\documentclass[11pt,a4paper]{article}

\usepackage[utf8]{inputenc}
\usepackage[T1]{fontenc}
\usepackage{amsmath, amssymb, amsthm}
\usepackage{mathtools}
\usepackage{bm}
\usepackage[margin=1in]{geometry}
\usepackage{enumitem}
\usepackage{booktabs}
\usepackage{hyperref}
\usepackage{cleveref}

\hypersetup{
    colorlinks=true,
    linkcolor=blue!70!black,
    citecolor=blue!70!black,
    urlcolor=blue!70!black
}

\newtheorem{definition}{Definition}
\newtheorem{proposition}{Proposition}
\newtheorem{remark}{Remark}

\DeclareMathOperator{\Cov}{Cov}
\DeclareMathOperator{\Corr}{Corr}
\DeclareMathOperator{\Var}{Var}
\DeclareMathOperator{\diag}{diag}

\title{Portfolio PnL Attribution and Risk Decomposition\\
Along Orthogonalized Stock Baskets}
\author{Research Notes}
\date{\today}

\begin{document}

\maketitle

\begin{abstract}
We present a mathematical framework for decomposing the performance and risk
of a long/short equity portfolio along a large number of long-only stock baskets.
Each basket is first orthogonalized against its corresponding market index to
remove systematic market exposure. Portfolio exposures to the orthogonalized
baskets are estimated via ridge regression. PnL is attributed additively across
baskets, and risk is decomposed using the Euler decomposition (MCTR/ACTR).
The framework scales to thousands of baskets.
\end{abstract}

\tableofcontents

% ======================================================================
\section{Setup and Notation}
\label{sec:setup}
% ======================================================================

\begin{definition}[Universe]
Let there be $N$ stocks indexed by $i = 1, \dots, N$, observed over
$T$ trading days indexed by $t = 1, \dots, T$. Let $B$ denote the number
of baskets indexed by $b = 1, \dots, B$.
\end{definition}

We use the following notation throughout:

\begin{center}
\begin{tabular}{cl}
\toprule
\textbf{Symbol} & \textbf{Description} \\
\midrule
$R_i(t)$ & Daily return of stock $i$ on day $t$ \\
$R^{\mathrm{mkt}}_m(t)$ & Daily return of market index $m$ on day $t$ \\
$R^{\mathrm{basket}}_b(t)$ & Daily return of basket $b$ on day $t$ \\
$R^{\mathrm{port}}(t)$ & Daily return of the portfolio on day $t$ \\
$\tilde{R}_b(t)$ & Orthogonalized return of basket $b$ on day $t$ \\
$\mathbf{R}(t)$ & $N$-vector of stock returns on day $t$ \\
$\bm{\omega}$ & $N$-vector of portfolio weights \\
$\mathbf{w}_b$ & Weight vector for basket $b$ (sparse, $N$-dimensional) \\
$\mathcal{S}_b$ & Set of stock indices belonging to basket $b$ \\
$m_b$ & Market index corresponding to basket $b$ \\
\bottomrule
\end{tabular}
\end{center}

% ======================================================================
\section{Stock Return Model}
\label{sec:stock_model}
% ======================================================================

Individual stock returns follow a single-factor model driven by the
market index:
\begin{equation}
\label{eq:stock_return}
R_i(t) = \alpha_i + \beta_i \, R^{\mathrm{mkt}}(t) + \epsilon_i(t),
\quad i = 1, \dots, N, \quad t = 1, \dots, T,
\end{equation}
where:
\begin{itemize}[nosep]
    \item $\alpha_i$ is the stock-specific drift,
    \item $\beta_i > 0$ is the market beta of stock $i$,
    \item $\epsilon_i(t) \sim \mathcal{N}(0, \sigma_i^2)$ is idiosyncratic noise,
          assumed independent across stocks and time.
\end{itemize}

\begin{remark}
In practice, betas are estimated or provided exogenously. The assumption
of a single market factor per region can be extended to multi-factor models
without changing the orthogonalization framework.
\end{remark}

% ======================================================================
\section{Basket Return Construction}
\label{sec:basket_construction}
% ======================================================================

\begin{definition}[Long-Only Basket]
Basket $b$ is defined by a set of constituents $\mathcal{S}_b \subset \{1, \dots, N\}$
and non-negative weights $w_{b,i} \geq 0$ for $i \in \mathcal{S}_b$, satisfying:
\begin{equation}
\sum_{i \in \mathcal{S}_b} w_{b,i} = 1.
\end{equation}
\end{definition}

The basket return is the weighted average of constituent returns:
\begin{equation}
\label{eq:basket_return}
R^{\mathrm{basket}}_b(t)
= \sum_{i \in \mathcal{S}_b} w_{b,i} \, R_i(t)
= \mathbf{w}_b^\top \mathbf{R}(t),
\end{equation}
where $\mathbf{w}_b \in \mathbb{R}^N$ is sparse with non-zero entries only for
$i \in \mathcal{S}_b$.

Each basket is associated with a market index via the mapping
$b \mapsto m_b$ (e.g., US baskets map to the S\&P\,500, European baskets
to the STOXX\,600).

% ======================================================================
\section{Portfolio Return}
\label{sec:portfolio}
% ======================================================================

The long/short portfolio has weights $\bm{\omega} \in \mathbb{R}^N$, where
$\omega_i > 0$ denotes a long position and $\omega_i < 0$ a short position.
For a dollar-neutral portfolio:
\begin{equation}
\sum_{i=1}^N \omega_i \approx 0.
\end{equation}

The portfolio return on day $t$ is:
\begin{equation}
\label{eq:portfolio_return}
R^{\mathrm{port}}(t) = \sum_{i=1}^N \omega_i \, R_i(t) = \bm{\omega}^\top \mathbf{R}(t).
\end{equation}

We define the gross leverage and net exposure as:
\begin{equation}
L_{\mathrm{gross}} = \sum_{i=1}^N |\omega_i|,
\qquad
L_{\mathrm{net}} = \sum_{i=1}^N \omega_i.
\end{equation}

% ======================================================================
\section{Orthogonalization of Basket Returns}
\label{sec:orthogonalization}
% ======================================================================

Raw basket returns are correlated with the market. To isolate the
idiosyncratic component, we orthogonalize each basket against its
corresponding market index.

\subsection{OLS Regression}

For each basket $b$ with market index $m_b$, we estimate via ordinary
least squares:
\begin{equation}
\label{eq:ols}
R^{\mathrm{basket}}_b(t)
= \hat{\alpha}_b + \hat{\beta}_b \, R^{\mathrm{mkt}}_{m_b}(t) + \hat{\varepsilon}_b(t).
\end{equation}

Define the regression matrix and response vector:
\begin{equation}
\mathbf{X}_{m_b}
= \begin{bmatrix} 1 & R^{\mathrm{mkt}}_{m_b}(1) \\
                   1 & R^{\mathrm{mkt}}_{m_b}(2) \\
                   \vdots & \vdots \\
                   1 & R^{\mathrm{mkt}}_{m_b}(T) \end{bmatrix}
\in \mathbb{R}^{T \times 2},
\qquad
\mathbf{y}_b
= \begin{bmatrix} R^{\mathrm{basket}}_b(1) \\
                   R^{\mathrm{basket}}_b(2) \\
                   \vdots \\
                   R^{\mathrm{basket}}_b(T) \end{bmatrix}
\in \mathbb{R}^T.
\end{equation}

The OLS solution is:
\begin{equation}
\label{eq:ols_solution}
\begin{bmatrix} \hat{\alpha}_b \\ \hat{\beta}_b \end{bmatrix}
= \left(\mathbf{X}_{m_b}^\top \mathbf{X}_{m_b}\right)^{-1}
  \mathbf{X}_{m_b}^\top \mathbf{y}_b.
\end{equation}

\subsection{Orthogonalized Returns}

\begin{definition}[Orthogonalized Basket Return]
The orthogonalized return of basket $b$ is the OLS residual:
\begin{equation}
\label{eq:ortho_return}
\tilde{R}_b(t) \equiv \hat{\varepsilon}_b(t)
= R^{\mathrm{basket}}_b(t) - \hat{\alpha}_b - \hat{\beta}_b \, R^{\mathrm{mkt}}_{m_b}(t).
\end{equation}
\end{definition}

\subsection{Diagnostic Statistics}

The coefficient of determination for basket $b$:
\begin{equation}
\label{eq:r_squared}
R^2_b = 1 - \frac{SS_{\mathrm{res},b}}{SS_{\mathrm{tot},b}}
= 1 - \frac{\displaystyle\sum_{t=1}^T \hat{\varepsilon}_b(t)^2}
           {\displaystyle\sum_{t=1}^T
            \left(R^{\mathrm{basket}}_b(t) - \bar{R}^{\mathrm{basket}}_b\right)^2},
\end{equation}
where $\bar{R}^{\mathrm{basket}}_b = \frac{1}{T}\sum_t R^{\mathrm{basket}}_b(t)$.

\begin{proposition}[Orthogonality]
By the normal equations of OLS, the residuals are uncorrelated with the
regressors:
\begin{equation}
\label{eq:orthogonality}
\Corr\!\left(\tilde{R}_b,\; R^{\mathrm{mkt}}_{m_b}\right) = 0
\quad \forall\; b = 1, \dots, B.
\end{equation}
\end{proposition}
\begin{proof}
The OLS first-order condition gives
$\mathbf{X}_{m_b}^\top \hat{\bm{\varepsilon}}_b = \mathbf{0}$,
which implies both $\sum_t \hat{\varepsilon}_b(t) = 0$ and
$\sum_t \hat{\varepsilon}_b(t) \, R^{\mathrm{mkt}}_{m_b}(t) = 0$.
The latter is the sample covariance (up to a constant), hence
the sample correlation is zero.
\end{proof}

% ======================================================================
\section{Exposure Estimation via Ridge Regression}
\label{sec:exposure}
% ======================================================================

We model the portfolio return as a linear combination of orthogonalized
basket returns:
\begin{equation}
\label{eq:exposure_model}
R^{\mathrm{port}}(t)
= \sum_{b=1}^B \delta_b \, \tilde{R}_b(t) + \eta(t),
\end{equation}
where $\delta_b$ is the portfolio's exposure to orthogonalized basket $b$
and $\eta(t)$ is the unexplained residual.

\subsection{Ridge Estimator}

When $B$ is large or the orthogonalized baskets are mutually correlated,
OLS is ill-conditioned. We use ridge regression (Tikhonov regularization)
with parameter $\lambda > 0$.

Define:
\begin{equation}
\tilde{\mathbf{X}}
= \bigl[\tilde{R}_b(t)\bigr]_{t,b} \in \mathbb{R}^{T \times B},
\qquad
\mathbf{y} = \bigl[R^{\mathrm{port}}(t)\bigr]_t \in \mathbb{R}^T.
\end{equation}

The ridge estimator is:
\begin{equation}
\label{eq:ridge}
\hat{\bm{\delta}}
= \left(\tilde{\mathbf{X}}^\top \tilde{\mathbf{X}}
  + \lambda \, \mathbf{I}_B\right)^{-1}
  \tilde{\mathbf{X}}^\top \mathbf{y}.
\end{equation}

\begin{remark}[Bayesian Interpretation]
The ridge estimator~\eqref{eq:ridge} is the maximum a posteriori (MAP)
estimate under the prior $\bm{\delta} \sim \mathcal{N}(\mathbf{0},
\lambda^{-1}\mathbf{I}_B)$ with Gaussian likelihood. Increasing $\lambda$
shrinks exposures toward zero.
\end{remark}

\subsection{In-Sample Fit}

The fitted values and residuals are:
\begin{equation}
\hat{\mathbf{y}} = \tilde{\mathbf{X}} \, \hat{\bm{\delta}},
\qquad
\hat{\bm{\eta}} = \mathbf{y} - \hat{\mathbf{y}}.
\end{equation}

The in-sample $R^2$:
\begin{equation}
\label{eq:model_r2}
R^2 = 1 - \frac{\|\hat{\bm{\eta}}\|^2}{\|\mathbf{y} - \bar{y}\,\mathbf{1}\|^2}
= 1 - \frac{\displaystyle\sum_t \hat{\eta}(t)^2}
           {\displaystyle\sum_t \left(R^{\mathrm{port}}(t) - \bar{R}^{\mathrm{port}}\right)^2}.
\end{equation}

% ======================================================================
\section{PnL Attribution}
\label{sec:pnl}
% ======================================================================

\subsection{Daily Attribution}

The daily PnL attributed to basket $b$ is:
\begin{equation}
\label{eq:daily_pnl}
\mathrm{PnL}_b(t) = \hat{\delta}_b \, \tilde{R}_b(t).
\end{equation}

The unexplained component absorbs the residual:
\begin{equation}
\label{eq:unexplained}
\mathrm{PnL}_{\mathrm{unexp}}(t)
= R^{\mathrm{port}}(t) - \sum_{b=1}^B \mathrm{PnL}_b(t).
\end{equation}

\begin{proposition}[Additive Decomposition]
The attribution is exact:
\begin{equation}
\label{eq:pnl_decomp}
R^{\mathrm{port}}(t)
= \sum_{b=1}^B \mathrm{PnL}_b(t) + \mathrm{PnL}_{\mathrm{unexp}}(t)
\quad \forall\; t.
\end{equation}
\end{proposition}

\subsection{Cumulative Attribution}

\begin{equation}
\label{eq:cum_pnl}
\mathrm{CumPnL}_b(\tau) = \sum_{t=1}^{\tau} \mathrm{PnL}_b(t),
\qquad \tau = 1, \dots, T.
\end{equation}

\subsection{Summary Statistics}

For each attribution source $s \in \{1, \dots, B, \mathrm{unexp}\}$, define:
\begin{align}
\label{eq:ann_return}
\text{Annualized Return}_s &= \bar{P}_s \times 252,
\quad \text{where } \bar{P}_s = \frac{1}{T}\sum_{t=1}^T \mathrm{PnL}_s(t), \\[6pt]
\label{eq:ann_vol}
\text{Annualized Vol}_s &= \hat{\sigma}(\mathrm{PnL}_s) \times \sqrt{252},
\quad \text{where } \hat{\sigma}(\mathrm{PnL}_s)
= \sqrt{\frac{1}{T-1}\sum_{t=1}^T \left(\mathrm{PnL}_s(t) - \bar{P}_s\right)^2}, \\[6pt]
\label{eq:sharpe}
\text{Sharpe}_s &= \frac{\text{Annualized Return}_s}{\text{Annualized Vol}_s}, \\[6pt]
\label{eq:pct_pnl}
\text{PnL\%}_s &= \frac{\displaystyle\sum_t \mathrm{PnL}_s(t)}
                       {\displaystyle\sum_t R^{\mathrm{port}}(t)} \times 100.
\end{align}

% ======================================================================
\section{Risk Decomposition}
\label{sec:risk}
% ======================================================================

We decompose portfolio risk along the orthogonalized baskets using the
Euler decomposition of a positively homogeneous risk measure.

\subsection{Covariance Matrix}

Let $\bm{\Sigma} \in \mathbb{R}^{B \times B}$ be the annualized covariance
matrix of orthogonalized basket returns:
\begin{equation}
\label{eq:cov_matrix}
\bm{\Sigma} = 252 \cdot \widehat{\Cov}\!\left(\tilde{\mathbf{R}}\right),
\end{equation}
where $\widehat{\Cov}$ denotes the sample covariance estimator.

\begin{remark}[Scaling Considerations]
For $B > 1000$, the sample covariance matrix becomes poorly conditioned
($T/B$ ratio too small). In practice, use shrinkage estimators
(e.g., Ledoit--Wolf), factor-based covariance, or sparse thresholding.
\end{remark}

\subsection{Portfolio Volatility}

The annualized portfolio volatility implied by the basket model is:
\begin{equation}
\label{eq:port_vol}
\sigma_p = \sqrt{\hat{\bm{\delta}}^\top \bm{\Sigma} \, \hat{\bm{\delta}}}.
\end{equation}

\subsection{Marginal Contribution to Total Risk (MCTR)}

\begin{definition}[MCTR]
The marginal contribution of basket $b$ to total portfolio risk is the
partial derivative of portfolio volatility with respect to the exposure:
\begin{equation}
\label{eq:mctr}
\mathrm{MCTR}_b
= \frac{\partial \sigma_p}{\partial \delta_b}
= \frac{(\bm{\Sigma} \, \hat{\bm{\delta}})_b}{\sigma_p}.
\end{equation}
\end{definition}

\begin{proof}
Differentiating $\sigma_p = (\hat{\bm{\delta}}^\top \bm{\Sigma}\,
\hat{\bm{\delta}})^{1/2}$ with respect to $\delta_b$:
\[
\frac{\partial \sigma_p}{\partial \delta_b}
= \frac{1}{2} \left(\hat{\bm{\delta}}^\top \bm{\Sigma}\,\hat{\bm{\delta}}\right)^{-1/2}
  \cdot 2 \, (\bm{\Sigma}\,\hat{\bm{\delta}})_b
= \frac{(\bm{\Sigma}\,\hat{\bm{\delta}})_b}{\sigma_p}.
\qedhere
\]
\end{proof}

\subsection{Absolute Contribution to Total Risk (ACTR)}

\begin{definition}[ACTR]
The absolute contribution of basket $b$ to total portfolio risk is:
\begin{equation}
\label{eq:actr}
\mathrm{ACTR}_b
= \hat{\delta}_b \times \mathrm{MCTR}_b
= \frac{\hat{\delta}_b \, (\bm{\Sigma} \, \hat{\bm{\delta}})_b}{\sigma_p}.
\end{equation}
\end{definition}

\begin{proposition}[Euler Decomposition]
ACTR provides an additive decomposition of portfolio volatility:
\begin{equation}
\label{eq:euler}
\sum_{b=1}^B \mathrm{ACTR}_b = \sigma_p.
\end{equation}
\end{proposition}
\begin{proof}
By Euler's theorem for homogeneous functions of degree one
($\sigma_p$ is homogeneous of degree one in $\hat{\bm{\delta}}$):
\[
\sum_{b=1}^B \hat{\delta}_b \, \frac{\partial \sigma_p}{\partial \delta_b}
= \sigma_p.
\]
Substituting $\mathrm{ACTR}_b = \hat{\delta}_b \, \mathrm{MCTR}_b$ yields
the result.
\end{proof}

\subsection{Percentage Risk Contribution}

\begin{equation}
\label{eq:risk_pct}
\mathrm{RiskPct}_b = \frac{\mathrm{ACTR}_b}{\sigma_p} \times 100,
\qquad
\sum_{b=1}^B \mathrm{RiskPct}_b = 100\%.
\end{equation}

% ======================================================================
\section{Summary of the Pipeline}
\label{sec:summary}
% ======================================================================

\begin{center}
\begin{tabular}{@{}clll@{}}
\toprule
\textbf{Step} & \textbf{Operation} & \textbf{Method} & \textbf{Output} \\
\midrule
1 & Basket returns & $\mathbf{w}_b^\top \mathbf{R}(t)$ & $R^{\mathrm{basket}}_b(t)$ \\
2 & Orthogonalize & OLS vs.\ market & $\tilde{R}_b(t)$, $\hat{\beta}_b$, $R^2_b$ \\
3 & Exposures & Ridge regression & $\hat{\bm{\delta}} \in \mathbb{R}^B$ \\
4 & PnL attribution & $\hat{\delta}_b \, \tilde{R}_b(t)$ & Daily/cumulative PnL \\
5 & Risk decomposition & MCTR/ACTR & $\mathrm{ACTR}_b$, $\mathrm{RiskPct}_b$ \\
\bottomrule
\end{tabular}
\end{center}

\subsection*{Computational Complexity}

\begin{itemize}[nosep]
\item \textbf{Orthogonalization}: $\mathcal{O}(B \cdot T)$ --- linear in the
      number of baskets.
\item \textbf{Ridge regression}: $\mathcal{O}(B^2 T + B^3)$ --- dominated by
      $\tilde{\mathbf{X}}^\top\tilde{\mathbf{X}}$ and the $B \times B$ solve.
\item \textbf{Risk decomposition}: $\mathcal{O}(B^2 T + B^2)$ --- covariance
      estimation and matrix-vector product.
\end{itemize}

For $B \gg T$, the ridge parameter $\lambda$ provides essential regularization,
and shrinkage covariance estimators should replace the sample covariance in
the risk decomposition.

\end{document}
